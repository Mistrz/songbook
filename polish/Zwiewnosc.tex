\tyt{Zwiewność}
\auth{sł. Bolesław Leśmian, muz. Zbigniew Stefański}

\begin{flushleft}
Brzęk muchy w pustym dzbanie, co stoi na półce \tab{a} \\
Smuga w oczach po znikłej za oknem jaskółce \tab{} \tab{G}\\
Cień ręki na murawie, a wszystko niczyje \tab{} \tab{F}\\
Ledwo się zazieleni, już ufa że żyje \tab{} \tab{E} \\
\vskip 3mm
Ref. A jak dumnie się modrzy u ciszy podnóża\\
\hspace{0.9cm}Jak buńczucznie do boju z mgłą się napurpurza \\
\hspace{0.9cm}A jest go tak niewiele, że mniej niż niebiesko \\
\hspace{0.9cm}Nic prócz tła, biały obłok z liliową przekreską \\
\vskip 3mm
Dal świata w ślepiach wróbla spotkanie traw z ciałem \\
Szmery w studni, ja w lesie, byłeś mgłą - bywałem \\
Usta twoje w alei, świt pod groblą, w młynie \\
Słońce w bramie na oścież, zgon pszczół w koniczynie \\
\vskip 3mm
Ref. A jak dumnie się modrzy u ciszy podnóża\\
\hspace{0.9cm}Jak buńczucznie do boju z mgłą się napurpurza \\
\hspace{0.9cm}A jest go tak niewiele, że mniej niż niebiesko \\
\hspace{0.9cm}Nic prócz tła, biały obłok z liliową przekreską \\
\vskip 3mm
Chód po ziemi człowieka, co na widnokresie \\
Malejąc mało zwiewną gęstwę ciała niesie \\
I w tej gęstwie się modli i gmatwa co chwila \\
I wyziera z gęstwy w świat i na motyla \\
\vskip 3mm
Ref. A jak dumnie się modrzy u ciszy podnóża\\
\hspace{0.9cm}Jak buńczucznie do boju z mgłą się napurpurza \\
\hspace{0.9cm}A jest go tak niewiele, że mniej niż niebiesko \\
\hspace{0.9cm}Nic prócz tła, biały obłok z liliową przekreską \\
\end{flushleft}
\clearpage
