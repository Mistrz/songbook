\tyt{Panna Kminkowa}
\auth{sł. i muz. Jerzy Reiser}

\begin{flushleft}
Obróciła się wiosna na pięcie \tab{}\tab{G D G}\\
Wyminęła się z latem we drzwiach \tab{C C$^{7+}$ D}\\
Wpatrywałaś się w okno zawzięcie \tab{G D G}\\
Może wyjdzie kominiarz na dach \tab{C C$^{7+}$ D G C$^7$ e}\\
Potem jechał gdzieś pociąg spóźniony \tab{e $H^7$ e}\\
Ktoś trzy asy wyłożył na stół \tab{}\tab{C C$^{9}_{7+}$ C$^6$}\\
A Ty kwiaty wkładałaś w wazony \tab{G C G}\\
I słyszałaś z daleka stuk kół \tab{}\tab{a C$^{9}_{7+}$ C D}\\
\vskip 3mm
A tymczasem rozgościł się lipiec \tab{G D G} \\
Na wesele zaprosił ze stu  \tab{}\tab{C C$^{9}_{7+}$ D}\\
Pasikonik pożyczył mu skrzypiec \tab{e H$^{7}$ e} \\
I w tym graniu tak biegłaś bez tchu  \tab{C D}\\
\vskip 3mm
Ref. Panno kminkowa, panno lipcowa \tab{}\tab{G D e D}\\
\hspace{0.9cm}Twoje są łąki, Twoje skowronki wszystkich łąk \tab{G G D D}\\
\hspace{0.9cm}Idziesz polami, w palcach łodyżkę masz \tab{}\tab{e CD G C}\\
\hspace{0.9cm}I gryziesz kminek, czarny przecinek nasz \tab{}\tab{G D G G}\\
\vskip 3mm
\hspace{0.9cm}Panna kminkowa, panna lipcowa \\
\hspace{0.9cm}Śmiać się gotowa letnia królowa kwietnych łąk \\
\hspace{0.9cm}Idzie polami, w palcach łodyżkę ma \\
\hspace{0.9cm}I gryzie kminek, czarny przecinek dnia \\
\vskip 3mm
Noce śniły, budziły się ranki \\
Koniczyny poczwórny był liść \\
Rwałaś groszek zielony przed gankiem \\
Kiedy przyszło po lato nam iść \\
Trwało lato i trwała muzyka \\
Rumieniłaś się wiśnią z mych ust \\
Las na wzgórzu horyzont zamykał \\
A Ty biegłaś ku niemu bez tchu \\
\vskip 3mm
Ref. Panno kminkowa, panno lipcowa\\
\hspace{0.9cm}Twoje są łąki, Twoje skowronki wszystkich łąk \\
\hspace{0.9cm}Idziesz polami, w palcach łodyżkę masz \\
\hspace{0.9cm}I gryziesz kminek, czarny przecinek nasz \\
\vskip 3mm
\hspace{0.9cm}Panna kminkowa, panna lipcowa \\
\hspace{0.9cm}Śmiać się gotowa letnia królowa kwietnych łąk \\
\hspace{0.9cm}Idzie polami, w palcach łodyżkę ma \\
\hspace{0.9cm}I gryzie kminek, czarny przecinek dnia \\
\vskip 3mm
A gdy lipiec się w drogę spakował \\
Oddał skrzypce i poszedł gdzieś w świat \\
Ty zostałaś mi, panno kminkowa \\
I łodyżki rzucane na wiatr \\
\vskip 3mm
Jesień deszczem Twe łąki przemoczy \\
Zima śniegiem okryje na mróz \\
A gdy lipiec zaglądnie Ci w oczy \\
Znowu będziesz tak biegła bez tchu \\
\vskip 3mm
Ref. Panno kminkowa, panno lipcowa\\
\hspace{0.9cm}Twoje są łąki, Twoje skowronki wszystkich łąk \\
\hspace{0.9cm}Idziesz polami, w palcach łodyżkę masz \\
\hspace{0.9cm}I gryziesz kminek, czarny przecinek nasz \\
\vskip 3mm
\hspace{0.9cm}Panna kminkowa, panna lipcowa \\
\hspace{0.9cm}Śmiać się gotowa letnia królowa kwietnych łąk \\
\hspace{0.9cm}Idzie polami, w palcach łodyżkę ma \\
\hspace{0.9cm}I gryzie kminek, czarny przecinek dnia \\
\end{flushleft}
\clearpage
