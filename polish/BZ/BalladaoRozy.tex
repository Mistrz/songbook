\tyt{Ballada o Róży}
\auth{sł. J.Koprowski, muz. Jerzy Reiser}

Raz w przezroczysty dzień słoneczny \tab{D h} \\
Jakiś nieboszczyk, czy przechodzień  \tab{G A}\\
Ubrany czarno, niedorzeczny \tab{}\tab{D h}\\
W kwitnącym znalazł się ogrodzie \tab{G A}\\
Przepłoszył dłonią ptasie śpiewy \tab{D A} \\
Zatoczył się od mocnych woni \tab{}\tab{h fis} \\
A kiedy róży dojrzał krzewy \tab{} \tab{G D}\\
Bardzo głęboko się ukłonił \tab{}\tab{D A D} \\
\hops
Patrzał i oto od spojrzenia \\
Gałązka pękiem strun zadrżała \\
I z ciszy liściastego cienia \\
Pąsowa nagle popatrzała \\
Dwa pąki dwojga piersi strzegły \\
Uśmiech, jak róża się rozwinął \\
Wargi gorąco krwią nabiegły \\
I rosę z płatków spił jak wino \\
\hops
Drapieżnie bronią kolce dzikie \\
Rozchylił płatki drżącej róży \\
I szorstkim liściem, czy językiem  \\
W mięsistość chłonnie się zanurzył \\
Trwało to dłużej od westchnienia  \\
A potem umknął niby złodziej \\
Zginął jak plama w światłocieniach  \\
Dziwny nieboszczyk, czy przechodzień