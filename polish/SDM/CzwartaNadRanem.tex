\tyt{Czarny blues o czwartej nad ranem}
\auth{sł. Adam Ziemianin, muz. Krzysztof Myszkowski}

Ref. Czwarta nad ranem \tab{}\tab{A} \\
\refr Może sen przyjdzie \tab{}\tab{cis}\\
\refr Może mnie odwiedzisz \tab{}\tab{D E A}\\
\refr Czwarta nad ranem \tab{}\tab{E}\\
\refr Może sen przyjdzie \tab{}\tab{fis}\\
\refr Może mnie odwiedzisz \tab{}\tab{D E A}\\
\hops
Czemu cię nie ma na odległość ręki? \tab{A E}\\
Czemu mówimy do siebie listami? \tab{fis cis}\\
Gdy ci to śpiewam - u mnie pełnia lata \tab{D A}\\
Gdy to usłyszysz - będzie środek zimy \tab{D E}\\ 
\hops
Czemu się budzę o czwartej nad ranem \tab{A E}\\
I włosy twoje próbuję ugłaskać \tab{fis cis}\\
Lecz nigdzie nie ma twoich włosów \tab{D A}\\
Jest tylko blada nocna lampka\tab{} \tab{D E}\\
Łysa śpiewaczka \tab{}\tab{fis}\\
\hops
Śpiewamy bluesa, bo czwarta nad ranem \\
Tak cicho, żeby nie zbudzić sąsiadów \\
Czajnik z gwizdkiem świruje na gazie \\
Myślałby kto, że rodem z Manhattanu \\
\hops
Ref. Czwarta nad ranem \\
\refr Może sen przyjdzie \\
\refr Może mnie odwiedzisz \\
\refr Czwarta nad ranem \\
\refr Może sen przyjdzie \\
\refr Może mnie odwiedzisz \\
\hops
Herbata czarna myśli rozjaśnia \\
A list twój sam się czyta \\
Że można go śpiewać \\
Za oknem mruczą bluesa \\
Topole z Krupniczej\\
\hops
I jeszcze strażak wszedł na solo \\
Ten z Mariackiej Wieży \\
Jego trąbka jak księżyc biegnie nad topolą \\
Nigdzie się jej nie spieszy \\
\hops
Ref. Już piąta\\
\refr Może sen przyjdzie \\
\refr Może mnie odwiedzisz \\
\refr Już piąta \\
\refr Może sen przyjdzie \\
\refr Może mnie odwiedzisz \\