\tyt{Bieszczadzkie Anioły}
\auth{sł. Adam Ziemianin, muz. Krzysztof Myszkowski}

Anioły są takie ciche\tab{}\tab{a} \\
Zwłaszcza te w Bieszczadach \tab{}\tab{G}\\
Gdy spotkasz takiego w górach\tab{}\tab{a} \\
Wiele z nim nie pogadasz \tab{}\tab{e}\\
Najwyżej na ucho ci powie \tab{}\tab{C G $G_{A-H}$}\\
Gdy będzie w dobrym humorze \tab{C F}\\
Że skrzydła nosi w plecaku \tab{\tab{C G}}\\
Nawet przy dobrej pogodzie \tab{}\tab{a G a}\\
\hops
Anioły są całe zielone \\
Zwłaszcza te w Bieszczadach \\
Łatwo w trawie się kryją \\
I w opuszczonych sadach \\
W zielone grają ukradkiem \\
Nawet karty mają zielone \\
Zielone mają pojęcie \\
A nawet zielony kielonek \\
\hops
Ref. Anioły bieszczadzkie, bieszczadzkie anioły \tab{a G a $\frac{a}{H}$} \\
\refr Dużo w was radości i dobrej pogody \tab{}\tab{C G C $\frac{C}{H}$}\\
\refr Bieszczadzkie anioły, anioły bieszczadzkie \tab{a G a $\frac{a}{H}$} \\
\refr Gdy skrzydłem cię trącą już jesteś ich bratem \tab{C G a}\\
\hops
Anioły są całkiem samotne \\
Zwłaszcza te w Bieszczadach \\
W kapliczkach zimą drzemią \\
Choć może im nie wypada \\
Czasem taki anioł samotny \\
Zapomni dokąd ma lecieć \\
I wtedy całe Bieszczady \\
Mają szaloną uciechę \\
\hops
Ref. Anioły bieszczadzkie, bieszczadzkie anioły \\
\refr Dużo w was radości i dobrej pogody \\
\refr Bieszczadzkie anioły, anioły bieszczadzkie \\
\refr Gdy skrzydłem cię trącą już jesteś ich bratem \\
\newpage
Anioły są wiecznie ulotne \\
Zwłaszcza te w Bieszczadach \\
Nas też czasami nosi \\
Po ich anielskich śladach \\
One nam przyzwalają \\
I skrzydłem wskazują drogę \\
I wtedy w nas się zapala \\
Wieczny bieszczadzki ogień \\
\hops
Ref. Anioły bieszczadzkie, bieszczadzkie anioły \\
\refr Dużo w was radości i dobrej pogody \\
\refr Bieszczadzkie anioły, anioły bieszczadzkie \\
\refr Gdy skrzydłem cię trącą już jesteś ich bratem \\
\hops
\refr Anioły bieszczadzkie, bieszczadzkie anioły \\
\refr Dużo w was radości i dobrej pogody \\
\refr Bieszczadzkie anioły, anioły bieszczadzkie \\
\refr Gdy skrzydłem cię musną już jesteś ich bratem