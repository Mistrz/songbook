\tyt{Tęsknica}
\auth{sł. i muz. Andrzej Wierzbicki }

Na przełęczy przysiadł wrzesień \tab{e D} \\
Śmieje się ukradkiem\tab{} \tab{C H7}\\
Skrzydłem kruka włosy czesze\tab{} \tab{e D} \\
Rozczochranym wiatrom \tab{}\tab{C H7 e} \\
Buczynie jej wargi sine\tab{} \tab{C G}\\
Maluje czerwienią \tab{}\tab{C H7} \\
I korale jarzębinie w bańki cerkwi leje \tab{e D C H7 e}\\
\hops
Ref. Do gór, do beskidzkich gór \tab{e C}\\
\refr Zawracamy kroki \tab{}\tab{D e}\\
\refr Przez równin zielony mur \tab{}\tab{e C} \\
\refr Dolin rzecznych krocie\tab{} \tab{D e} \\
\refr Do gór, do beskidzkich gór \tab{e C}\\
\refr Zawracamy oczy \tab{}\tab{D H7}\\
\refr By dojrzeć w buczyny pniach \tab{e C} \\
\refr Madonn twarze złote\tab{} \tab{H7 e} \\
\hops
Mgły strącając po dolinach \\
Jesień wozem jedzie \\
Znarowione konie spina \\
Worek chleba wiezie \\
I naszym wołaniem \\
Zmęczona odchodzi \\
Tylko echo wyprowadza \\
Na rozstajne drogi  \\
\hops
Ref. Do gór, do beskidzkich gór\\
\refr Zawracamy kroki \\
\refr Przez równin zielony mur \\
\refr Dolin rzecznych krocie \\
\refr Do gór, do beskidzkich gór \\
\refr Zawracamy oczy \\
\refr By dojrzeć w buczyny pniach \\
\refr Madonn twarze złote