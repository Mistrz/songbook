\tyt{Beskid}
\auth{sł. i muz. Andrzej Wierzbicki}

\begin{flushleft}
A w Beskidzie rozzłocony buk, \tab{}\tab{G C D G}\\
A w Beskidzie rozzłocony buk. \tab{}\tab{G C G a D }\\
Będę chodził Bukowiną z dłutem w ręku,\tab{} \tab{C D G}\\
By w dziewczęcych twarzach uśmiech rzeźbić, \tab{}\tab{C G}\\
Niech nie płaczą już, \tab{}\tab{}\tab{C $\frac{C}{H}$ a D}\\
Niech się cieszą po kapliczkach moich dróg. \tab{}\tab{C D G} \\
\vskip 3mm
Ref. Beskidzie, malowany cerkiewny dach,\tab{}\tab{}\tab{G C D G} \\
\hspace{0.9cm}Beskidzie, zapach miodu w bukowych pniach. \tab{}\tab{G C H7 e}\\
\hspace{0.9cm}Tutaj wracam, gdy ruda jesień na przełęcze swój tobół niesie. \tab{C D G C}\\
\hspace{0.9cm}Słucham bicia dzwonów w przedwieczorny czas. \tab{}\tab{G C $\frac{C}{H}$ a D}\\
\hspace{0.9cm}Beskidzie, malowany wiatrami dom,  \tab{}\tab{}\tab{G C D G}\\
\hspace{0.9cm}Beskidzie, tutaj słowa inaczej brzmią, \tab{}\tab{}\tab{G C H7 e}\\
\hspace{0.9cm}Kiedy krzyczę w jesienną ciszę, kiedy wiatrem szeleszczą liście, \tab{C D G C} \\
\hspace{0.9cm}Kiedy wolność się tuli w ciepło moich rąk \tab{}\tab{G C $\frac{C}{H}$ a D}\\
\hspace{0.9cm}Gdy, jak źrebak się tuli do mych rąk \tab{}\tab{}\tab{C D G}\\
\vskip 3mm
A w Beskidzie zamyślony czas \\
A w Beskidzie zamyślony czas \\
Będę chodził z nim poddaszem gór \\
By zerwanych marzeń struny \\
Przywiązywać w niespokojne dłonie drzew \\
Niech mi grają na rozstajach moich dróg \\
\vskip 3mm
Ref. Beskidzie, malowany cerkiewny dach, \\
\hspace{0.9cm}Beskidzie, zapach miodu w bukowych pniach. \\
\hspace{0.9cm}Tutaj wracam, gdy ruda jesień na przełęcze swój tobół niesie. \\
\hspace{0.9cm}Słucham bicia dzwonów w przedwieczorny czas. \\
\hspace{0.9cm}Beskidzie, malowany wiatrami dom, \\
\hspace{0.9cm}Beskidzie, tutaj słowa inaczej brzmią, \\
\hspace{0.9cm}Kiedy krzyczę w jesienną ciszę, kiedy wiatrem szeleszczą liście, \\
\hspace{0.9cm}Kiedy wolność się tuli w ciepło moich rąk \\
\hspace{0.9cm}Gdy, jak źrebak się tuli do mych rąk \\
\end{flushleft}
\clearpage
