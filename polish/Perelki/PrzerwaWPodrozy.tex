\tyt{Przerwa w podróży}
\auth{sł. i muz. Andrzej Wierzbicki}

Przy piwie w karczmie w Limanowej \tab{}\tab{a H7}\\
Zapatrzeni w siwe mgły jesienne\tab{}\tab{E a}\\
Czekaliśmy na autobusowe \tab{}\tab{}\tab{F C}\\
Ostatnie lata połączenie\tab{} \tab{}\tab{B E}\\
Bóg przez okno złoty talar rzucił \tab{}\tab{a d}\\
Słońce na obrusie \tab{}\tab{}\tab{G C}\\
Od dziewczyny przy barze pożyczyłem uśmiech \tab{a d}\\
Oddam w autobusie \tab{}\tab{}\tab{E a}\\
\hops
Ref. A po lesie wiatr (a po lesie wiatr) \tab{$a_{A-H-C-A}$ H7 E} \\
\refr Rwie na strzępy pajęczyny nić \tab{a} \\
\refr A po polu wiatr (a po polu wiatr) \tab{$a_{A-H-C-A}$ H7 e} \\
\refr Rozsypuje kopce siana w pył \tab{a A7} \\
\refr Do puszystych traw się tuli \tab{d} \\
\refr Skrada się do pustych ptasich gniazd \tab{G C $\frac{C}{H}$ a}  \\
\refr Po strumieniach z wodą śpiewa \tab{d E} \\
\refr Lekkomyślny wiatr \tab{}\tab{a}\\
\hops
Wpatrzeni w okno milczeliśmy wszyscy \\
Nikt nie przerywał nam czekania \\
Nawet gitary zacisnęły zęby \\
Wiedziały - już nie będzie grania \\
Oczy dziewczyny szeptały - zostań \\
Czas się zatrzymał w Limanowej \\
Oddałem uśmiech, przewróciłem kufel \\
Na stole talar spał \\
\hops
Ref. A po lesie wiatr (a po lesie wiatr) \\
\refr Rwie na strzępy pajęczyny nić  \\
\refr A po polu wiatr (a po polu wiatr)  \\
\refr Rozsypuje kopce siana w pył  \\
\refr Do puszystych traw się tuli  \\
\refr Skrada się do pustych ptasich gniazd  \\
\refr Po strumieniach z wodą śpiewa  \\
\refr Lekkomyślny wiatr