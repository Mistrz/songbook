\tyt{Ballada o Świętym Mikołaju}
\auth{sł. i muz. Andrzej Wierzbicki}

W rozstrzelanej chacie \tab{}\tab{a G E} \\
Rozpaliłem ogień \tab{}\tab{a G a} \\
Z rozwalonych pieców \tab{}\tab{a G E}\\
Pieśń wyniosłem węgle \tab{}\tab{F E}\\
Naciągnąłem na drzazgi gontów \tab{a $\frac{a}{H}$ C}\\
Błękitną płachtę nieba \tab{}\tab{G E}\\
Będę malował od nowa \tab{}\tab{a d C $\frac{C}{H}$ a}\\
Wioskę w dolinie \tab{}\tab{d E a ~$\frac{a}{H}$}\\
\hops
Ref. Święty Mikołaju  \tab{}\tab{C G}\\
\refr Opowiedz jak tu było \tab{}\tab{C E E7}\\
\refr Jakie pieśni śpiewano \tab{}\tab{a d C $\frac{C}{H}$ a}\\
\refr Gdzie się pasły konie \tab{}\tab{d E a G}\\
\refr Święty Mikołaju \tab{}\tab{C G}\\
\refr Opowiedz jak tu było \tab{}\tab{C E E7}\\
\refr Jakie pieśni śpiewano \tab{}\tab{a d C $\frac{C}{H}$ a} \\
\refr Gdzie się pasły konie \tab{}\tab{d E}\\
\hops
A on nie chce gadać \\
Ze mną po polsku \\
Z wypalonych źrenic \\
Tylko deszcze płyną \\
Hej, ślepcze nauczę swoje \\
Dziecko po łemkowsku \\
Będziecie razem żebrać \\
W malowanych wioskach \\
\hops
Ref. Święty Mikołaju \\
\refr Opowiedz jak tu było \\
\refr Jakie pieśni śpiewano \\
\refr Gdzie się pasły konie \\
\refr Święty Mikołaju \\
\refr Opowiedz jak tu było \\
\refr Jakie pieśni śpiewano \\
\refr Gdzie się pasły konie