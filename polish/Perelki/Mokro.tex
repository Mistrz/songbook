\tyt{Mokro (Ślimak)}
\auth{sł. Barbara Sobolewska, muz. L. Małecki}
\begin{flushleft}
Napadał deszcz do szklanki \tab{}\tab{G7+ a}\\
napadał w bród  \tab{}\tab{C9/5 hE}\\
W deszczówce moczą nogi \tab{}\tab{a D}\\
zmęczone cienie chudych psów \tab{}\tab{a D G7+}\\
\hops
Z sufitu kapią krople\tab{}\tab{G7+ a}\\
w dziurawy dzban \tab{}\tab{C9/5 hE}\\
Zagapił się przez szybę\tab{}\tab{a D}\\
na mokrą panią mokry pan. \tab{}\tab{a D G}\\
\hops
Ref. W przemoczonej trawie \tab{}\tab{G h e7}\\
\hspace{0.9cm}Chrapie ślimak zły \tab{}\tab{F C E7 a}\\
\hspace{0.9cm}Ślimaku pokaż rogi\tab{}\tab{D G C}\\
\hspace{0.9cm}dam ci sera na pierogi \tab{}\tab{D G}\\
\hspace{0.9cm}Nie pokażę rogów\tab{}\tab{G h e7}\\
\hspace{0.9cm}bo nakapie mi na lewy róg i prawy \tab{F C E7 a D G C}\\
\hspace{0.9cm}Nie nie wyjdę z mojej trawy. \tab{D G G7+}\\
\hops
Kałuże tyją strasznie, pękają w szwach\\
Wyleją na chodniki i będzie powódź, że aż strach\\
\hops
W kaloszu masz dwie dziury, w płaszczu też\\
I będziesz siedział u mnie, aż wyschną dachy, zniknie deszcz. \\
\hops
Ref. W przemoczonej trawie ...
\end{flushleft}
\newpage