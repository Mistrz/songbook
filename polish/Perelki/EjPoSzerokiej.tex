\tyt{Ej, po szerokiej drodze (Flora)}
\auth{sł. Konstanty Ildefons Gałczyński, muz. B. Szmigielski}

\tab{}\tab{}\tab{D $D_{Fis-G-A-G-Fis-E-D}$ G C2 G x2}\\
Ej! po szerokiej drodze, sam jeden, sam jeden z gitarą  \tab{D G h E $E_{E-Fis-G}$}\\
To znowu nie jest tak źle... \tab{}\tab{}\tab{G A G}\\
Cienie w czapkach złodziejskich pod parkanami się gonią \tab{D G h E $E_{E-Fis-G}$}\\
A sosny kołyszą się. \tab{}\tab{}\tab{G A D}\\
\hops
Ref. Do pieca, Floro, podkładaj, a ty wiosnę opowiadaj \tab{D e fis G fis}\\
\refr Jeszcze trwożną, sancta muza \tab{}\tab{e D A}\\
\refr Mnie wiersz, tobie wichura, muzo moja złotopióra \tab{D e fis G fis} \\
\refr Chwalę twe liściaste zgłoski \tab{}\tab{e A G D }\\
\hops
Śpiewaj bracie, sam jeden, uderzaj w należne struny \\
Nuta po nucie... \\
Gwiazdy! A ty z gwiazdami jak z dziewczynami się bawisz \\
W zaczarowanej tancbudzie. \\
\hops
Ref. Do pieca, Floro, podkładaj, a ty wiosnę opowiadaj \\
\refr Jeszcze trwożną, sancta muza \\
\refr Mnie wiersz, tobie wichura, muzo moja złotopióra \\
\refr Chwalę twe liściaste zgłoski \\
\hops
Piosenki, które układasz, wrzuć w księżyc, jak na poczcie \\
Ktoś otrzyma list. \\
A resztę już w niebie załatwią. Bo reszta w tę noc majową \\
Nie znaczy nic.  \\
\hops
Ref. Do pieca, Floro, podkładaj, a ty wiosnę opowiadaj \\ 
\refr Jeszcze trwożną, sancta muza \\
\refr Mnie wiersz, tobie wichura, muzo moja złotopióra \\
\refr Chwalę twe liściaste zgłoski \\
\hops
\refr Do pieca, Floro, podkładaj, a ty wiosnę opowiadaj \\
\refr Jeszcze trwożną, sancta muza \\
\refr Mnie wiersz, tobie wichura, muzo moja złotopióra \\
\refr Chwalę twe liściaste zgłoski 