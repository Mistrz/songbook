\tyt{Ballada o Cześku piekarzu}
\auth{sł. i muz. Wojtek Bellon}
\begin{flushleft}
Chleba takiego jak ten od Cześka \tab{D A}\\
Nie kupisz nigdzie, nawet w Warszawie \tab{e G Fis h}\\
Bo Czesiek Piekarz nie piekł, lecz tworzył \tab{G A}\\
Bochny, jak z mąki słoneczne kołacze. \tab{D A}\\
\hop
Kłaniali mu się ludzie, gdy wyjrzał \tab{D A}\\
Przez okno w kitlu łyknąć powietrza \tab{e G Fis h}\\
A kromkę masłem smarujac każdy \tab{G A}\\
Mówił: Nad chleby ten chleb od Cześka. \tab{D A D}\\
\hops
Ref. Chleb się chlebie, chleb się chlebie \tab{C G e a}\\
\hspace{0.9cm}Bo nad chleb być może co \tab{C h e}\\
\hspace{0.9cm}Chleb się chlebie, chleb się chlebie \tab{C G e a}\\
\hspace{0.9cm}Niech ci nigdy nie zabraknie \tab{C}\\
\hspace{0.9cm}Drożdży, wody, rąk i ziarna \tab{D A e h}\\
\hspace{0.9cm}(mruczał Czesiek tak noc w noc) \tab{GD A G D}\\
\hops
A o porankach chlebem pachnących\\
Gdy pora idzie spać na piekarzy\\
Zaczerwienione przymykał oczy\\
Czesiek i siadał z dłutem przy stole\\
\hop
Ciągle te same włosy i trochę\\
Za duży nos, w drzewie cierpliwym\\
Pieściły ręce dziesiątki razy\\
W poranki świeżym chlebem pachnące.\\
\hops
Ref. Chleb się chlebie...
\hops
Nikt takich słów jak miasto miastem\\
Nie znał i "źle się dzieje" mówili\\
Na obraz czerniał Czesiek razowca\\
Kruszał podobnie bułce zleżałej\\
\hop
Gdy go znaleźli na piasku z wojska\\
Dłuto jak wbite w bochen miał w garści\\
I  nie wie nikt co Cześka wzięło\\
Lecz śpiewa każdy jak miasto miastem\\
\hops
Ref. Chleb się chlebie
\end{flushleft}
\newpage