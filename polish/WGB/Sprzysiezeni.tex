\tyt{Sprzysiężeni}
\auth{sł. i muz. Wojciech Bellon}

Sprzysiężeni budząc się świtem \tab{a G D} \\
Przykrywają palcami oczy \tab{}\tab{F C E}\\
By zatrzymać chociaż przez chwilę \tab{a G D} \\
Nić wysnutą z osnowy nocy \tab{}\tab{C E F G} \\
Nić, co nieba barwą się mieniąc \tab{C G}\\
Diretissimę w ścianie kreśli \tab{}\tab{C G} \\
Potem dnia zakładają brzemię \tab{}\tab{h A} \\
I ruszają w drogę ku szczęściu \tab{}\tab{C E7 a}\\
\hops
Ref. Mija dzień, koło się toczy \tab{}\tab{a C G E}\\
\refr Marzeniami kładą się cienie \tab{a C d F}\\
\refr I odradza się każdej nocy \tab{}\tab{C E} \\
\refr I odradza się każdej nocy \tab{}\tab{a C d F}\\
\refr Sprzysiężenie górskiego kamienia \tab{C E a} \\
\hops
Sprzysiężeni - przyjazne dłonie \\
Plotą węzeł nad ogniem watry \\
I wpatrzeni w gasnący płomień \\
Nucą pieśni pachnące wiatrem \\
Nie rozplotą ni burze, ni waśnie \\
Tego, co złączone przez ogień \\
Słońce wokół - wciąż jaśniej i jaśniej \\
Zakwitł kamień dziś górskim głogiem \\
\hops
Ref. Mija dzień, koło się toczy\\
\refr Marzeniami kładą się cienie \\
\refr I odradza się każdej nocy \\
\refr I odradza się każdej nocy \\
\refr Sprzysiężenie górskiego kamienia \\
\hops
A gdy wiatr sprzysiężonym w oczy zawieje \\
Bliski uśmiech w cień nocy odejdzie \\
Bukowina opuszcza ramiona \\
Bukowina łeb pochyla siwy \\
Czas odpływa, z czasem smutek kona \\
Lecz wspomnienia pozostają żywe \\
\hops
Ref. Mija dzień, koło się toczy...
