\phantomsection
\tyt{Bukowina II}
\auth{sł. i muz. Wojciech Bellon }

Dość wytoczyli bań próżnych przed domy kalecy, \tab{C d F C}\\
Żyją, jak żyli, bezwolni, głusi i ślepi. \tab{}\tab{C d F C}\\
Nie współczuj, szkoda łez i żalu, \tab{}\tab{d G C $\frac{C}{H}$ a}\\
Bezbarwni są, bo chcą być szarzy. \tab{}\tab{d G C $\frac{C}{H}$ a}\\
Ty wyżej, wyżej bądź i dalej \tab{}\tab{}\tab{e F Fis G C}\\
Niż ci, co się wyzbyli marzeń. \tab{}\tab{}\tab{d G C}\\
\hops
Ref. Niechaj zalśni Bukowina w barwie malin, \tab{}\tab{C F G $G_{G-G-A-H}$}\\
\refr Niechaj zabrzmi Bukowina w wiatru szumie,\tab{C F G} \\
\refr Dzień minął, dzień minął, nadszedł wieczór, \tab{C d C}\\
\refr Świece gwiazd zapalił, \tab{}\tab{}\tab{F G}\\
\refr Siadł przy ogniu, pieśń posłyszał i umilkł.\tab{C d F C} \\
\hops
Po dniach zgiełkliwych, po nocach wyłożonych brukiem \\
W zastygłym szkliwie gwiazd neonowych próżno szukać \\
Tego, co tylko zielonością \\
Na palcach zaplecionych drzemie. \\
Rozewrzyj dłonie mocniej, mocniej \\
Za kark chwyć słońce, sięgnij w niebo \\
\hops
Ref. Niechaj zalśni Bukowina w barwie malin, \\
\refr Niechaj zabrzmi Bukowina w wiatru szumie, \\
\refr Dzień minął, dzień minął, nadszedł wieczór, \\
\refr Świece gwiazd zapalił, \\
\refr Siadł przy ogniu, pieśń posłyszał i umilkł. \\
\hops
Odnaleźć musisz, gdzie chmury górom dłoń podają, \\
Gdzie deszcz i susza, gdzie lipce, październiki, maje \\
Stają się rokiem, węzłem życia, \\
Swój dom bukowy, zawieszony \\
U nieba pnia, kroplą żywicy, \\
Błękitny, złoty i zielony \\
\hops
Ref. Niechaj zalśni Bukowina w barwie malin, \\
\refr Niechaj zabrzmi Bukowina w wiatru szumie, \\
\refr Dzień minął, dzień minął, nadszedł wieczór, \\
\refr Świece gwiazd zapalił, \\
\refr Siadł przy ogniu, pieśń posłyszał i umilkł.
