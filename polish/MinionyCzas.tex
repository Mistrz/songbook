\tyt{Miniony czas}
\auth{sł. i muz. Krzysztof Libionko}

\begin{flushleft}
Dwadzieścia lat to wiek wspaniały \tab{C F}\\
To złoty wiek jak sądzi świat \tab{}\tab{G C}\\
Gdzieś z drugiej strony pola chwały \tab{C F}\\
Poległo mych dwadzieścia lat \tab{}\tab{G C}\\
To były pieskie chwile, owszem \tab{C F}\\
Lecz przeszły w mig, jak z bicza trzasł \tab{G C}\\
I sam dziś sobie ich zazdroszczę \tab{C F}\\
Umarł już, nie wróci ten czas \tab{}\tab{G $F7^+$ $F_{F-E-D}$ C}\\
\vskip 3mm
Ref. Nie ma ni rys, ni skaz miniony czas \tab{}\tab{d G}\\
\hspace{0.9cm}I gdy wyciągną swe kopyta \tab{}\tab{C $\frac{C}{H}$ a}\\
\hspace{0.9cm}Wybaczyć można tym, co nie zranią już nas \tab{d G C C7 (d G $F_{F-E-D}$ C)}\\
\vskip 3mm
Nie mówmy źle o nieboszczykach \\
Czy w twej pamięci małej gąski \\
Kołacze się wśród zgranych kart \\
Ów as daremnych gier miłości \\
W zatęchłym łóżku się dusiła \\
By nam powiedzieć wkrótce pas \\
A jednak z łezką ją wspominasz \\
Piękną dziś, cóż płomień jej zgasł \\
\vskip 3mm
Ref. Nie ma ni rys, ni skaz miniony czas\\
\hspace{0.9cm}I gdy wyciągną swe kopyta \\
\hspace{0.9cm}Wybaczyć można tym, co nie zranią już nas\\
\vskip 3mm
Nie dziwne więc że niczym płaczce \\
Rozdartych mi nie szkoda szat \\
Gdy stary szkielet w czarnej paczce \\
Odprowadzam na tamten świat \\
Choć najpodlejsza to kanalia \\
Jaką nosiła ziemia ta \\
Od płaczu niech to was nie zwalnia \\
Sczezła i czysta jest jak łza \\
\vskip 3mm
Ref. Nie ma ni rys, ni skaz miniony czas\\
\hspace{0.9cm}I gdy wyciągną swe kopyta \\
\hspace{0.9cm}Wybaczyć można tym, co nie zranią już nas \\
\end{flushleft}
\clearpage
