\tyt{Herr Mannelig}
\auth{sł. i muz. trad. szewdzkie, jęz. szwedzki}
\begin{flushleft}

Bittida en morgon innan solen upprann\\
Innan foglarna började sjunga\\
Bergatrollet friade till fager ungersven\\
Hon hade en falskeliger tunga\\
\hops
\hspace{0.9cm}Herr Mannelig herr Mannelig trolofven i mig\\
\hspace{0.9cm}För det jag bjuder så gerna\\
\hspace{0.9cm}I kunnen väl svara endast ja eller nej\\
\hspace{0.9cm}Om i viljen eller ej
\hops
Eder vill jag gifva de gångare tolf\\
Som gå uti rosendelunde\\
Aldrig har det varit någon sadel uppå dem\\
Ej heller betsel uti munnen\\
\hops
\hspace{0.9cm}Herr Mannelig...
\hops
Eder vill jag gifva de qvarnarna tolf\\
Som stå mellan Tillö och Ternö\\
Stenarna de äro af rödaste gull\\
Och hjulen silfverbeslagna\\
\hops
\hspace{0.9cm}Herr Mannelig...
\hops
Eder vill jag gifva ett förgyllande svärd\\
Som klingar utaf femton guldringar\\
Och strida huru I strida vill\\
Stridsplatsen skolen i väl vinna\\
\hops
\hspace{0.9cm}Herr Mannelig...
\hops
Eder vill jag gifva en skjorta så ny\\
Den bästa I lysten att slita\\
Inte är hon sömnad av nål eller trå\\
Men virkat av silket det hvita\\
\hops
\hspace{0.9cm}Herr Mannelig...
\hops
Sådana gåfvor toge jag väl emot\\
Om du vore en kristelig qvinna\\
Men nu så är du det värsta bergatroll\\
Af Neckens och djävulens stämma\\
\hops
\hspace{0.9cm}Herr Mannelig...
\hops
Bergatrollet ut på dörren sprang\\
Hon rister och jämrar sig svåra\\
Hade jag fått den fager ungersven\\
Så hade jag mistat min plåga\\
\hops
\hspace{0.9cm}Herr Mannelig...
\end{flushleft}
\newpage