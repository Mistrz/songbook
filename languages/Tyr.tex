\tyt{Torsteins kvæði}
\auth{Týr, jęz. farerski}

Vilja tit lýða og ljóð geva mær, eg bróti av bragdartátti \tab{a E}\\
Kongur ráddi for Nøríki, hann tógva synir átti \tab{a C F}
\hops
\refr Rennur og rennur foli mín \tab{d a}\\
\refr Grønari grund og vín bar reyða lund \tab{E a C}\\
\refr Stíg at dansa stund \tab{}\tab{C a}\\
\refr Kátur leikar foli mín \tab{}\tab{d F}\\
\refr Á grønari grund \tab{}\tab{E a E}
\hops
Átt hevur hann sær tógva synir, báðar kann eg væl nevna\\
Magnus og hann Torstein jall, teir kunna væl
dreingjum stevna
\hops
\refr Rennur og rennur foli mín...
\hops
Jallurin gongur for kongin inn og sigur honum frá\\
Nú er Torstein, sonur tín, kvittaður londum frá
\hops
\refr Rennur og rennur foli mín...
\hops
Um hann vil av ríkinum fara, eingin skal honum banna\\
Hann kann ei síni forløg flý, sjálvur má hann tað sanna
\hops
\refr Rennur og rennur foli mín...
\hops
Tað lovaði eg bæði faðir og móðir, tá eg í vøggu lá\\
Eg skyldi ei ræðast tann heita eld, ei heldur tað hvassa stál
\hops
\refr Rennur og rennur foli mín...
\hops
Vargurin liggur í vetrarmjøll, staddur í stórari neyð\\
So er tann maður, í víggi stendur, næstan tolir deyð
\hops
\refr Rennur og rennur foli mín...
\hops
Frúgvin var bæði studd og leidd, inn í sín faðirs veldi\\
Ljómin skein av akslagrein, tað skyggir av tignareldi
\hops
\refr Rennur og rennur foli mín...
\hops
Nú er hetta kvæðið endað, tað man góðum gegna\\
Torstein tók við ríkjunum baðum so við hvøru tegna
\hops
\refr Rennur og rennur foli mín...

\clearpage
\tyt{Ormurin Langi}
\auth{Týr, jęz. farerski}

\sng{1.}\\
Viljið tær hoyra kvæði mítt,\\
vilja tær orðum trúgva,\\
um hann Ólav Trygvason,\\
hagar skal ríman snúgva.
\hops
\refr Glymur dansur í høll,\\
\refr Dans sláið ring\\
\refr Glaðir ríða Noregs menn\\
\refr til Hildar ting.
\hops
\sng{3.}\\
Knørrur var gjørdur á Noregs landi,\\
gott var í honum evni:\\
sjútti alin og fýra til,\\
var kjølurin millum stevna.
\hops
\refr Glymur dansur í høll...
\hops
\sng{8.}\\
Har kom maður á bergið oman\\
við sterkum boga í hendi:\\
"Jallurin av Ringaríki\\
hann meg higar sendi."
\hops
\refr Glymur dansur í høll...
\hops
\sng{10.}\\
"Einar skalt tú nevna meg,\\
væl kann boga spenna,\\
Tambar eitur mín menski bogi,\\
ørvar drívur at renna."
\hops
\refr Glymur dansur í høll...
\hops
\sng{11.}\\
"Hoyr tú tað, tú ungi maður,\\
vilt tú við mær fara,\\
tú skalt vera mín ørvargarpur\\
Ormin at forsvara."
\hops
\refr Glymur dansur í høll...
\newpage
\sng{12.}\\
Gingu teir til strandar oman,\\
ríkir menn og reystir,\\
lunnar brustu og jørðin skalv:\\
teir drógu knørr úr neysti.
\hops
\refr Glymur dansur í høll...
\hops
\sng{71.}\\
Einar spenti triðja sinni,\\
æt1ar jall at raka,\\
tá brast strongur av stáli stinna,\\
í boganum tók at braka.
\hops
\refr Glymur dansur í høll...
\hops
\sng{72.}\\
Allir hoyrdu streingin springa,\\
kongurin seg undrar:\\
"Hvat er tað á mínum skipi,\\
so ógvuliga dundrar ?"
\hops
\refr Glymur dansur í høll...
\hops
\sng{73.}\\
Svaraði Einar Tambarskelvir\\
kastar boga sín\\
"Nú brast Noregi úr tínum hondum,\\
kongurin, harri mín !"
\hops
\refr Glymur dansur í høll...
\hops
\sng{30.}\\
Nú skal lætta ljóðið av,\\
eg kvøði ei longur á sinni;\\
nú skal eg taka upp annan tátt;\\
dreingir leggi í minni !
\hops
\refr Glymur dansur í høll...

\clearpage
\tyt{Sigurdkvade}
\auth{Krauka, jęz. duński}

Ormen gled af guldet frem,\\
det skal hver mand vide.\\
Sigurd sad på Granes ryg\\
Dristig mon han ride
\hops
\refr Grane bar guld af hede,\\
\refr Grane bar guld af hede,\\
\refr Sigurd svinger sværdet i vrede.\\
\refr Sigurd over ormen vandt,\\
\refr Grane bar guldet af heden.\\
\hops
Tredive alen dybt var vandet,\\
hvorved ormen lå,\\
brystet raged' op deraf,\\
På fjeldet halen lå.
\hops
Rask da var unge Sigurd,\\
han med sværdet slog,\\
kløvede den blanke orm\\
Sønder i stykker to.
\hops
Ormens hjerte stegte han,\\
det gik nok så trangt:\\
spiddet af det hårde træ\\
Var tredive alen langt.
\hops
Sigurd blev om hænder hed,\\
strøg sig da om munden:\\
fuglesang og dyresprog\\
Straks han fatte kunne.
\hops
Fafner's rige skat af guld\\
Sigurd monne få,\\
da han vog den blanke orm\\
På Gnitahede lå.
\hops
I den årle morgenstund\\
førend sol oprandt,\\
fir'ogtyve kister guld\\
På Granes ryg han bandt 