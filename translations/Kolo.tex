\tyt{Koło (Will The Circle Be Unbroken)}
\auth{sł. i muz. tradycyjna piosenka amerykańska\\
tłum. Emil Staniucha, Jano Mościcki}
\begin{flushleft}
Stałem długo przy mym oknie, \tab{G G7}\\
Chmurny, chłodny to był dzień \tab{C G}\\
Wtem karawan się wytoczył, \tab{}\tab{e G}\\
Aby matkę mą zabrać hen! \tab{}\tab{G D G}\\
\hops
Ref. Jeśli koło to wytrzyma, \tab{}\tab{G}\\
\hspace{0.9cm}Wkrótce Panie, wkrótce już,\tab{C G}\\
\hspace{0.9cm}Lepszy dom tam czeka na nią,\tab{e G}\\
\hspace{0.9cm}W niebie, w niebie o Panie mój!\tab{G D G}\\
\hops
Rzekłem do karawaniarza:\\
„Ruszaj wolno, nie śpiesz się”.\\
Przy tej pani, którą wieziesz,\\
Dłużej wtedy będę szedł.\\
\hops
Ref. Jeśli koło to wytrzyma,\\
\hspace{0.9cm}Wkrótce Panie, wkrótce już,\\
\hspace{0.9cm}Lepszy dom tam czeka na nią,\\
\hspace{0.9cm}W niebie, w niebie o Panie mój!\
\hops
Szedłem blisko, tuż za trumną\\
Dusząc w sobie przypływ łez,\\
Lecz nie mogłem ukryć bólu,\\
Gdy musiała grobie lec.
\hops
Ref. Jeśli koło to wytrzyma,\\
\hspace{0.9cm}Wkrótce Panie, wkrótce już,\\
\hspace{0.9cm}Lepszy dom tam czeka na nią,\\
\hspace{0.9cm}W niebie, w niebie o Panie mój!\\
\end{flushleft}
\newpage